\chapter{Conclusion}

\section{Bilan du stage}
Ce stage de quatre mois chez Tech Solutions m'a permis de participer activement au développement complet d'une application web moderne, depuis l'analyse des besoins jusqu'au déploiement en production. L'application développée valide les principes d'architecture prônés par \cite{martin2017clean} et \cite{fielding2000rest}.

\subsection{Apports professionnels}
\begin{itemize}
	\item Maîtrise du développement full-stack avec des technologies modernes incluant React \cite{react2024} et Node.js \cite{nodejs2024}
	\item Expérience concrète des méthodologies Agile
	\item Gestion de projet dans un environnement professionnel
	\item Collaboration au sein d'une équipe de développement
\end{itemize}

\subsection{Apports personnels}
Sur le plan personnel, ce stage m'a permis de :
\begin{itemize}
	\item Développer mon autonomie et ma prise d'initiative
	\item Améliorer mes compétences en communication technique
	\item Apprendre à gérer les contraintes temps/réalité
\end{itemize}

\section{Perspectives d'évolution}
L'application développée dispose d'un fort potentiel d'évolution, s'appuyant sur des fondations techniques solides inspirées des travaux de \cite{martin2017clean} :

\subsection{Améliorations techniques}
\begin{itemize}
	\item Migration vers une architecture microservices
	\item Implémentation de WebSockets pour le temps réel
	\item Application mobile React Native
\end{itemize}

\subsection{Fonctionnalités futures}
\begin{itemize}
	\item Intelligence artificielle pour la prédiction des délais
	\item Intégration avec d'autres outils (Slack, GitHub, etc.)
	\item Marketplace d'extensions
\end{itemize}

\section{Retour d'expérience}
Cette expérience professionnelle a été extrêmement enrichissante et a confirmé mon attrait pour le développement d'applications web. Elle m'a également permis d'identifier des domaines où je souhaite approfondir mes connaissances, notamment en architecture cloud et en DevOps.

Le succès de ce projet démontre l'importance d'une approche centrée utilisateur et d'une stack technique bien choisie, combinant React \cite{react2024} pour le frontend et Node.js \cite{nodejs2024} pour le backend. Les principes architecturaux de \cite{martin2017clean} et les standards REST de \cite{fielding2000rest} se sont avérés déterminants pour la réussite du projet.

Je suis convaincu que les compétences acquises durant ce stage me seront précieuses pour ma future carrière dans le développement logiciel.