\chapter{Introduction}

\section{Contexte du stage}
Le stage s'est déroulé au sein de Tech Solutions, une entreprise innovante spécialisée dans le développement de solutions logicielles pour les PME. Fondée en 2015, l'entreprise compte aujourd'hui une équipe de 25 développeurs et consultants techniques.

La digitalisation des processus métier étant devenue un enjeu crucial pour la compétitivité des entreprises, Tech Solutions a identifié un besoin croissant pour des outils de gestion de projet adaptés aux spécificités des petites et moyennes entreprises. Comme le souligne \cite{martin2017clean}, une architecture bien conçue est essentielle pour répondre à ces défis.

\section{Problématique}
Les solutions existantes sur le marché présentent souvent les limitations suivantes:
\begin{itemize}
	\item Complexité d'utilisation pour les petites équipes
	\item Fonctionnalités superflues augmentant les coûts
	\item Manque de flexibilité pour les workflows spécifiques
	\item Interfaces peu intuitives
\end{itemize}

Ces limitations vont à l'encontre des principes d'architecture REST énoncés par \cite{fielding2000rest}, qui privilégient la simplicité et l'évolutivité.

\section{Objectifs du stage}
Les objectifs principaux de ce stage étaient :
\begin{enumerate}
	\item Analyser les besoins spécifiques des PME en matière de gestion de projet
	\item Concevoir une architecture technique scalable et maintenable en s'appuyant sur les bonnes pratiques de \cite{martin2017clean}
	\item Développer une application web full-stack répondant à ces besoins en utilisant React \cite{react2024} et Node.js \cite{nodejs2024}
	\item Mettre en production et assurer le déploiement de l'application
\end{enumerate}

\section{Méthodologie adoptée}
Le développement a suivi une méthodologie Agile avec des sprints de deux semaines. Chaque sprint incluait :
\begin{itemize}
	\item Planification des fonctionnalités
	\item Développement et tests unitaires
	\item Revue de code en équipe
	\item Déploiement sur l'environnement de test
\end{itemize}

L'approche technique s'est inspirée des principes de \cite{martin2017clean} pour l'architecture et des standards REST de \cite{fielding2000rest} pour la conception des API.