\chapter{Analyses Statistiques Avancées}

\section{Tests d'hypothèses}

\subsection{Test e Student}
Pour comparer les moyennes de deux groupes indépendants, nous utilisons le test t de Student:

\begin{equation}
	t = \frac{\bar{X}_1 - \bar{X}_2}{s_q \sqrt{\frac{1}{n_1} + \frac{1}{n_2}}}
	\label{eq:student-test}
\end{equation}
ou $s_p$ est l'écart-type poolé : 
\begin{equation}
	s_p = \sqrt{\frac{(n_1-1)s_1^2 + (n_2-1)s_2^2}{n_1 + n_2 - 2}}
	\label{eq:pooled-std}
\end{equation}

\subsection{Analyse de puissance statistique}
La puissance du test est donnée par :
\begin{equation}
	\text{Puissance} = 1 - \beta = \Phi\left(z_{1-\alpha/2} - \frac{\delta}{\sigma\sqrt(2/n)}\right) + \Phi\left(-z_{1-\alpha/2} - \frac{\delta}{\sigma\sqrt{2/n}}\right)
	\label{eq:statistical-power}
\end{equation}

\begin{table}[H]
	\centering
	\caption{Résultats des tests statistiques comparatifs}
	\begin{tabular}{lrrrrrr}
		\toprule
		\textbf{Comparaison} & \textbf{t-value} & \textbf{ddl} & \textbf{p-value} & \textbf{IC 95\%} & \textbf{Size} & \textbf{Power} \\
		\midrule
		Algo A vs Algo B & 3.45 & 58 & 0.0012 & [0.124, 0.456] & 0.89 & 0.92 \\
		Algo A vs Algo C & 2.12 & 58 & 0.0387 & [0.023, 0.289] & 0.56 & 0.67 \\
		Algo B vs Algo C & 1.78 & 58 & 0.0801 & [-0.015, 0.234] & 0.45 & 0.52 \\
		Notre vs Meilleur & 4.67 & 58 & 0.0001 & [0.234, 0.567] & 1.23 & 0.98 \\
		\bottomrule
	\end{tabular}
	\label{tab:statistical-tests}
\end{table}


\section{Analyses multivariées}
\subsection{Analyse en composantes principales}
la protection sur composantes principales s'écrit :
\begin{equation}
	Z = XW
	\label{eq:pca-projection}
\end{equation}
ou $w$ est la matrice des vecteurs propres de la matrice de covariance.

\subsection{Équation de l'ellipse de confiance}
Pour une distribution normale bivariée :

\begin{equation}
	\frac{(x-\mu_x)^2}{\sigma_x^2} + \frac{(y-\mu)^2}{\sigma_y^2} - 2\rho\frac{(x-\mu_x)(y-\mu_y)}{\sigma_x\sigma_y} = -2\log(1-p)
	\label{er:confidence-ellipse}
\end{equation}

\begin{figure}[H]
	\centering
	\begin{tikzpicture}
		\begin{axis}[
			width=0.8\textwidth,
			height=0.6\textwidth,
			xlabel={Premiere composante principale (68.2\%)},
			ylabel={Deuxieme Composante principale (23.7\%)},
			legend pos=north east,
			grid=major
		]
		
		\addplot[only marks, mark=*, blue] coordinates {
			(1.2, 0.8) (1.5, 0.9) (1.1, 0.7) (1.3, 0.6) (1.4, 0.8)
			(1.6, 0.5) (1.2, 0.9) (1.5, 0.7) (1.3, 0.8) (1.4, 0.6)
		};
		
		\addplot[only marks, mark=square*, red] coordinates {
			(-0.8, -0.5) (-1.1, -0.6) (-0.9, -0.4) (-1.0, -0.7) (-0.7, -0.5)
			(-1.2, -0.8) (-0.8, -0.6) (-1.1, -0.7) (-0.9, -0.5) (-1.0, -0.6)
		};
		
		\draw[blue, thick, rotate around={45:(1.35, 073)}] (1.35, 0.73) ellipse (0.4 and 0.2);
		\draw[red, thick, rotate around={30:(-0.95, -0.59)}] (-0.95, -0.59) ellipse (0.3 and 0.15);
		
		\legend{Groupe 1, Groupe 2}
		\end{axis}
	\end{tikzpicture}
	
	\caption{Analyse de composantes principales avec ellipses de confiance}
	\label{fig:pca-analysis}
\end{figure}

\section{Série temporelles et prévisions}
\subsection{Modèle ARIMA}
Le modèle ARIMA(p,d,q) s'écrit : 
\begin{equation}
	(1 - \sum_{i=1}^p \phi_iB^i)(1 - B)^d X_t = (1 + \sum_{i=1}^q \theta_i B^i) \epsilon_t 
	\label{eq:arima-model}
\end{equation}

ou $B$ est l'opérateur de retard.

\subsection{Équation de prévision}

La prévision à l'horizon $h$ est donnée par : 

\begin{equation}
	\hat{X}_{t+h} = \mathbb{E}[X_{t+h} | X_1, \dots, X_t]
	\label{eq:forest-equation}
\end{equation}

\begin{table}[H]
	\centering
	\caption{Performance des modèle de prévision}
	\begin{tabular}{lrrrrr}
		\toprule
		\textbf{Modèle} & \textbf{RMSE} & \textbf{MAE} & \textbf{MAPE (\%)} & \textbf{R²} & \textbf{AIC} \\
		\midrule
		ARIMA(1,1,1) & 23.45 & 18.67 & 4.23 & 0.892 & 1256.78 \\
		SARIMA(1,1,1)(1,1,1,12) & 18.92 & 15.23 & 3.45 & 0.934 & 1189.45 \\
		ETS (A,N,N) & 25.67 & 20.45 & 4.67 & 0.867 & 1345.23 \\
		Prophet & 20.12 & 16.78 &  3.89 & 0.915 & 1223.56 \\
		Notre modèle & \textbf{15.34} & \textbf{12.45} & \textbf{2.89} & \textbf{0.956} & \textbf{1123.67} \\
		\bottomrule
	\end{tabular}
	\label{tab:forest-performance}
\end{table}




































